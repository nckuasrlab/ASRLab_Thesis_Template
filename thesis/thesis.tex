% ------------------------------------------------
% 常見論文內容次序:
% 1.口試合格證明(初稿不需要)
% 2.中英文摘要(論文以中文撰寫者須附英文延伸摘要)
% 3.誌謝(初稿不需要)
% 4.目錄
% 5.表目錄
% 6.圖目錄
% 7.符號
% 8.主文
% 9.參考文獻
% 10.附錄
%
% 註: 參考文獻書寫注意事項:
% (1).
%    文學院之中文文獻依分類及年代順序排列。
%    其他學院所之文獻依英文姓氏第一個字母
%    (或中文姓氏第一個字筆劃)及年代順序排列。
% (2).
%    期刊文獻之書寫依序為:
%        姓名、文章名稱、期刊名、卷別、期別、頁別、年代。
% (3).
%    書寫之文獻依序為:
%        姓名、書名、出版商名、出版地、頁別、年代。
% ------------------------------------------------

% ------------------------------------------------
%          載入基本設定 Basic configuration
% ------------------------------------------------
% Don't need to modify this section.
\input{template/configure}
\begin{document}

% ------------------------------------------------
%                封面內頁 Inner Cover
% ------------------------------------------------
\singlespacing 
\newpage
\phantomsection
\thispagestyle{empty}
\newgeometry{top=2.3cm,bottom=3cm,left=2cm,right=2cm,nohead,nofoot}
  
\begin{center}

\begin{minipage}[c][5cm][t]{\textwidth}
  \begin{center}
    \makebox[10cm][s]{\Huge 國立成功大學}\vspace{1cm}		\\
    \makebox[8cm][s]{\Huge 資訊工程研究所}\vspace{1cm}		\\
    \makebox[5cm][s]{\Huge 碩士論文}\vspace{1cm}			\\
    \makebox[5cm][c]{\Huge (初稿)}
  \end{center}
\end{minipage}

\vspace{5.5cm}

\begin{minipage}[c][5cm][t]{\textwidth}
  \begin{center}
    \makebox[\textwidth][c]{\parbox{\paperwidth}
    {\center \Large 國立成功大學碩博士畢業論文\\LaTex模板}}
    \vspace{0.5cm}\\
    \makebox[\textwidth][c]{\parbox{\paperwidth}
    {\center \Large National Cheng Kung University (NCKU)\\Thesis/Dissertation Template in LaTex}}
  \end{center}
\end{minipage}

\vspace{1.0cm}

\begin{minipage}[c][4.5cm][t]{\textwidth}
  \begin{center}
    {
      % -------------Student-------------
      \hspace{2.0em}
      \makebox[4.0em][r]{\Large 學生:}
      \makebox[6.0em][l]{\Large 你的名字}
      \makebox[8.0em][c]{}
      \makebox[4.0em][r]{\Large Student:}
      \makebox[8.0em][l]{\Large Your Name}
      \vspace{0.5cm}\\
      % -------------Advisor-------------
      \hspace{2.0em}
      \makebox[4.0em][r]{\Large 指導老師:}
      \makebox[6.0em][l]{\Large A博士}
      \makebox[8.0em][c]{}
      \makebox[4.0em][r]{\Large Advisor:}
      \makebox[8.0em][l]{\Large Dr. A}
      \vspace{0.1cm}\\
      % -------------Co-Advisor-1-------------
      \hspace{2.0em}
      \makebox[4.0em][r]{\Large 共同指導:}
      \makebox[6.0em][l]{\Large B博士}
      \makebox[8.0em][c]{}
      \makebox[4.0em][r]{\Large Co-Advisor:}
      \makebox[8.0em][l]{\Large Dr. B}
      \vspace{0.1cm}\\
      % -------------Co-Advisor-2-------------
      \hspace{1.7em}
      \makebox[4.0em][r]{}
      \makebox[6.0em][l]{\Large C博士}
      \makebox[8.0em][c]{}
      \makebox[4.0em][r]{}
      \makebox[8.0em][l]{\Large Dr. C}
      \\
    }
  \end{center}
\end{minipage}

\vspace{0.5cm}

\makebox[8cm][s]{\Large 中華民國105年12月}

\end{center}
\restoregeometry
\clearpage
\setstretch{1.2}

% ------------------------------------------------
%  學位考試論文證明書 Defense Certificate (初稿不需要)
% ------------------------------------------------
\newpage
\phantomsection
\thispagestyle{empty}
\includepdf[pages=-]{context/oral/oral_example.pdf}
\clearpage

% ------------------------------------------------
%                  摘要 Abstract
% ------------------------------------------------
% 除了外籍生, 本地生和僑生都要同時編寫中文和英文摘要
% 論文以中文撰寫須以英文補寫 800 至 1200 字數的英文延伸摘要 (Extended Abstract)
% 詳細可看附件的學校要求或看example中的英文延伸摘要

% -------------中文摘要-------------
\setcounter{page}{1}
\pagenumbering{roman}
\newpage
\phantomsection
\chapter*{摘要}
\pagestyle{plain}

除了外籍生, 本地生和僑生都是要編寫中文和英文摘要. 論文以中文撰寫須以英文補寫 800 至 1200 字數的英文延伸摘要 (Extended Abstract), 如需要的話請修改`./context/abstract/extended.tex'.

外籍生則可免填中文摘要, 而在上傳時網頁資訊需填上`NONE'.

\par{\noindent \bf 關鍵字:}{關鍵字1, 關鍵字2, 關鍵字3, 關鍵字4}
\clearpage
\setstretch{1.2}

% -------------English Abstract-------------
\newpage
\phantomsection
\chapter*{Abstract}
\pagestyle{plain}

Write your abstract here.

\par{\noindent \bf Keyword:}{ Keyword1, Keyword2, Keyword3, Keyword4}
\clearpage
\setstretch{1.2}


% ------------------------------------------------
%           誌謝 Acknowledgments(初稿不需要)
% ------------------------------------------------
\newpage
\phantomsection
\chapter*{誌謝}
\pagestyle{plain}

在這邊寫你的感謝 (對父母, 老師, 同學, 朋友等的感謝).

\clearpage
\setstretch{1.2}


% ------------------------------------------------
%              目錄 Index of contents
% ------------------------------------------------
% 目錄有三種,分為內容、表格及圖片索引

% -------------內容索引-------------
\renewcommand*\contentsname{Table of Contents}
\singlespacing
\newpage
\phantomsection
\pagestyle{plain}
\tableofcontents
\clearpage
\setstretch{1.2}

% -------------表格索引-------------
\listoftables
% -------------圖片索引-------------
\listoffigures

% ------------------------------------------------
%              Nomenclature(optional)
% ------------------------------------------------
\newpage
\phantomsection
\chapter*{Nomenclature}
\pagestyle{plain}

\begin{center}
  \begin{tabular}{C{0.2\textwidth} C{0.4\textwidth}}
    \hline
    \underline{Symbol} & \centerline{\underline{Description}} \\
    $\alpha$	&	Symbol of alpha 	\\
    $\beta$		&	Beta				\\
    $\gamma$	&	Gamma				\\
    \hline
  \end{tabular}
\end{center}

\EmptyLine

\begin{center}
  \begin{tabular}{C{0.2\textwidth} C{0.4\textwidth} C{0.35\textwidth}}
    \hline
    \underline{Symbol} & \underline{Meaning} & \underline{SI unit of measure}	\\
    $g$		&	Standard gravity	&	$9.80665 m/s^2$							\\
    $c$		&	Speed of light		&	$\approx3.00\times108 m/s$				\\
    $l$		&	Length				&	meter (m)								\\
 	\hline
  \end{tabular}
  \vspace{0.1cm}\\{List of common physics notations}
\end{center}
\clearpage
\setstretch{1.2}


% ------------------------------------------------
%                   Introduction
% ------------------------------------------------
\setcounter{page}{1}
\pagenumbering{arabic}
\newpage
\phantomsection
\chapter{Introduction}
\pagestyle{plain}

Write your introduction here.

\clearpage
\setstretch{1.2}


% ------------------------------------------------
%                    Objective
% ------------------------------------------------
% The objective of the paper is the reason given for writing the paper.

\newpage
\phantomsection
\chapter{Objective}
\pagestyle{plain}

Write your objective here.

\clearpage
\setstretch{1.2}

% ------------------------------------------------
%                  Related Work
% ------------------------------------------------
\newpage
\phantomsection
\chapter{Related Work}
\pagestyle{plain}

Write your relatd work here.

\clearpage
\setstretch{1.2}

% ------------------------------------------------
%                   Conclusion
% ------------------------------------------------
\newpage
\phantomsection
\chapter{Conclusion}
\pagestyle{plain}

Write your conclusion here.

\clearpage
\setstretch{1.2}

% ------------------------------------------------
%                  \Cite Example
% ------------------------------------------------
\newpage
\phantomsection
\chapter{Cite Example}
\pagestyle{plain}

This chapter shows how to use $\backslash$cite command. The $\backslash$cite command read the \textbf{.bib file} to find the references.

\begin{enumerate}
  \item{point 1 \cite{test1}}
  \item{point 2 \cite{test2}}
    \begin{enumerate}
      \item{point 2-1 \cite{test3}}
      \item{point 2-2}
    \end{enumerate}
  \item{point 3 \cite{test4}}
\end{enumerate}


\clearpage
\setstretch{1.2}

% ------------------------------------------------
%               參考文獻 References
% ------------------------------------------------
% References and bibliography

% Import the files that contain your references.
% If you set some references file,
% you need to use at least one cite to make Latex work.
\singlespacing
\newpage
\phantomsection
\renewcommand\bibname{References}	% references title

\bibliographystyle{ieeetr}			% reference style
\bibliography{thesis}				% target .bib file

\clearpage
\setstretch{1.2}

% ------------------------------------------------
%                   附錄 Appendix
% ------------------------------------------------
\BeginAppendixChapter				% User-defined function
\SetupAppendixNumberingFormat		% User-defined function
\newpage
\phantomsection
\chapter{Title of Appendix A}

Write your Appendix here.

  \section{Dolor sit amet}
    Appendix

\clearpage
\setstretch{1.2}

\end{document}